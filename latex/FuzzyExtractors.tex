\documentclass[11pt]{article}

    \usepackage[breakable]{tcolorbox}
    \usepackage{parskip} % Stop auto-indenting (to mimic markdown behaviour)
    

    % Basic figure setup, for now with no caption control since it's done
    % automatically by Pandoc (which extracts ![](path) syntax from Markdown).
    \usepackage{graphicx}
    % Maintain compatibility with old templates. Remove in nbconvert 6.0
    \let\Oldincludegraphics\includegraphics
    % Ensure that by default, figures have no caption (until we provide a
    % proper Figure object with a Caption API and a way to capture that
    % in the conversion process - todo).
    \usepackage{caption}
    \DeclareCaptionFormat{nocaption}{}
    \captionsetup{format=nocaption,aboveskip=0pt,belowskip=0pt}

    \usepackage{float}
    \floatplacement{figure}{H} % forces figures to be placed at the correct location
    \usepackage{xcolor} % Allow colors to be defined
    \usepackage{enumerate} % Needed for markdown enumerations to work
    \usepackage{geometry} % Used to adjust the document margins
    \usepackage{amsmath} % Equations
    \usepackage{amssymb} % Equations
    \usepackage{textcomp} % defines textquotesingle
    % Hack from http://tex.stackexchange.com/a/47451/13684:
    \AtBeginDocument{%
        \def\PYZsq{\textquotesingle}% Upright quotes in Pygmentized code
    }
    \usepackage{upquote} % Upright quotes for verbatim code
    \usepackage{eurosym} % defines \euro

    \usepackage{iftex}
    \ifPDFTeX
        \usepackage[T1]{fontenc}
        \IfFileExists{alphabeta.sty}{
              \usepackage{alphabeta}
          }{
              \usepackage[mathletters]{ucs}
              \usepackage[utf8x]{inputenc}
          }
    \else
        \usepackage{fontspec}
        \usepackage{unicode-math}
    \fi

    \usepackage{fancyvrb} % verbatim replacement that allows latex
    \usepackage{grffile} % extends the file name processing of package graphics
                         % to support a larger range
    \makeatletter % fix for old versions of grffile with XeLaTeX
    \@ifpackagelater{grffile}{2019/11/01}
    {
      % Do nothing on new versions
    }
    {
      \def\Gread@@xetex#1{%
        \IfFileExists{"\Gin@base".bb}%
        {\Gread@eps{\Gin@base.bb}}%
        {\Gread@@xetex@aux#1}%
      }
    }
    \makeatother
    \usepackage[Export]{adjustbox} % Used to constrain images to a maximum size
    \adjustboxset{max size={0.9\linewidth}{0.9\paperheight}}

    % The hyperref package gives us a pdf with properly built
    % internal navigation ('pdf bookmarks' for the table of contents,
    % internal cross-reference links, web links for URLs, etc.)
    \usepackage{hyperref}
    % The default LaTeX title has an obnoxious amount of whitespace. By default,
    % titling removes some of it. It also provides customization options.
    \usepackage{titling}
    \usepackage{longtable} % longtable support required by pandoc >1.10
    \usepackage{booktabs}  % table support for pandoc > 1.12.2
    \usepackage{array}     % table support for pandoc >= 2.11.3
    \usepackage{calc}      % table minipage width calculation for pandoc >= 2.11.1
    \usepackage[inline]{enumitem} % IRkernel/repr support (it uses the enumerate* environment)
    \usepackage[normalem]{ulem} % ulem is needed to support strikethroughs (\sout)
                                % normalem makes italics be italics, not underlines
    \usepackage{mathrsfs}
    

    
    % Colors for the hyperref package
    \definecolor{urlcolor}{rgb}{0,.145,.698}
    \definecolor{linkcolor}{rgb}{.71,0.21,0.01}
    \definecolor{citecolor}{rgb}{.12,.54,.11}

    % ANSI colors
    \definecolor{ansi-black}{HTML}{3E424D}
    \definecolor{ansi-black-intense}{HTML}{282C36}
    \definecolor{ansi-red}{HTML}{E75C58}
    \definecolor{ansi-red-intense}{HTML}{B22B31}
    \definecolor{ansi-green}{HTML}{00A250}
    \definecolor{ansi-green-intense}{HTML}{007427}
    \definecolor{ansi-yellow}{HTML}{DDB62B}
    \definecolor{ansi-yellow-intense}{HTML}{B27D12}
    \definecolor{ansi-blue}{HTML}{208FFB}
    \definecolor{ansi-blue-intense}{HTML}{0065CA}
    \definecolor{ansi-magenta}{HTML}{D160C4}
    \definecolor{ansi-magenta-intense}{HTML}{A03196}
    \definecolor{ansi-cyan}{HTML}{60C6C8}
    \definecolor{ansi-cyan-intense}{HTML}{258F8F}
    \definecolor{ansi-white}{HTML}{C5C1B4}
    \definecolor{ansi-white-intense}{HTML}{A1A6B2}
    \definecolor{ansi-default-inverse-fg}{HTML}{FFFFFF}
    \definecolor{ansi-default-inverse-bg}{HTML}{000000}

    % common color for the border for error outputs.
    \definecolor{outerrorbackground}{HTML}{FFDFDF}

    % commands and environments needed by pandoc snippets
    % extracted from the output of `pandoc -s`
    \providecommand{\tightlist}{%
      \setlength{\itemsep}{0pt}\setlength{\parskip}{0pt}}
    \DefineVerbatimEnvironment{Highlighting}{Verbatim}{commandchars=\\\{\}}
    % Add ',fontsize=\small' for more characters per line
    \newenvironment{Shaded}{}{}
    \newcommand{\KeywordTok}[1]{\textcolor[rgb]{0.00,0.44,0.13}{\textbf{{#1}}}}
    \newcommand{\DataTypeTok}[1]{\textcolor[rgb]{0.56,0.13,0.00}{{#1}}}
    \newcommand{\DecValTok}[1]{\textcolor[rgb]{0.25,0.63,0.44}{{#1}}}
    \newcommand{\BaseNTok}[1]{\textcolor[rgb]{0.25,0.63,0.44}{{#1}}}
    \newcommand{\FloatTok}[1]{\textcolor[rgb]{0.25,0.63,0.44}{{#1}}}
    \newcommand{\CharTok}[1]{\textcolor[rgb]{0.25,0.44,0.63}{{#1}}}
    \newcommand{\StringTok}[1]{\textcolor[rgb]{0.25,0.44,0.63}{{#1}}}
    \newcommand{\CommentTok}[1]{\textcolor[rgb]{0.38,0.63,0.69}{\textit{{#1}}}}
    \newcommand{\OtherTok}[1]{\textcolor[rgb]{0.00,0.44,0.13}{{#1}}}
    \newcommand{\AlertTok}[1]{\textcolor[rgb]{1.00,0.00,0.00}{\textbf{{#1}}}}
    \newcommand{\FunctionTok}[1]{\textcolor[rgb]{0.02,0.16,0.49}{{#1}}}
    \newcommand{\RegionMarkerTok}[1]{{#1}}
    \newcommand{\ErrorTok}[1]{\textcolor[rgb]{1.00,0.00,0.00}{\textbf{{#1}}}}
    \newcommand{\NormalTok}[1]{{#1}}

    % Additional commands for more recent versions of Pandoc
    \newcommand{\ConstantTok}[1]{\textcolor[rgb]{0.53,0.00,0.00}{{#1}}}
    \newcommand{\SpecialCharTok}[1]{\textcolor[rgb]{0.25,0.44,0.63}{{#1}}}
    \newcommand{\VerbatimStringTok}[1]{\textcolor[rgb]{0.25,0.44,0.63}{{#1}}}
    \newcommand{\SpecialStringTok}[1]{\textcolor[rgb]{0.73,0.40,0.53}{{#1}}}
    \newcommand{\ImportTok}[1]{{#1}}
    \newcommand{\DocumentationTok}[1]{\textcolor[rgb]{0.73,0.13,0.13}{\textit{{#1}}}}
    \newcommand{\AnnotationTok}[1]{\textcolor[rgb]{0.38,0.63,0.69}{\textbf{\textit{{#1}}}}}
    \newcommand{\CommentVarTok}[1]{\textcolor[rgb]{0.38,0.63,0.69}{\textbf{\textit{{#1}}}}}
    \newcommand{\VariableTok}[1]{\textcolor[rgb]{0.10,0.09,0.49}{{#1}}}
    \newcommand{\ControlFlowTok}[1]{\textcolor[rgb]{0.00,0.44,0.13}{\textbf{{#1}}}}
    \newcommand{\OperatorTok}[1]{\textcolor[rgb]{0.40,0.40,0.40}{{#1}}}
    \newcommand{\BuiltInTok}[1]{{#1}}
    \newcommand{\ExtensionTok}[1]{{#1}}
    \newcommand{\PreprocessorTok}[1]{\textcolor[rgb]{0.74,0.48,0.00}{{#1}}}
    \newcommand{\AttributeTok}[1]{\textcolor[rgb]{0.49,0.56,0.16}{{#1}}}
    \newcommand{\InformationTok}[1]{\textcolor[rgb]{0.38,0.63,0.69}{\textbf{\textit{{#1}}}}}
    \newcommand{\WarningTok}[1]{\textcolor[rgb]{0.38,0.63,0.69}{\textbf{\textit{{#1}}}}}


    % Define a nice break command that doesn't care if a line doesn't already
    % exist.
    \def\br{\hspace*{\fill} \\* }
    % Math Jax compatibility definitions
    \def\gt{>}
    \def\lt{<}
    \let\Oldtex\TeX
    \let\Oldlatex\LaTeX
    \renewcommand{\TeX}{\textrm{\Oldtex}}
    \renewcommand{\LaTeX}{\textrm{\Oldlatex}}
    % Document parameters
    % Document title
    \title{FuzzyExtractors}
    
    
    
    
    
% Pygments definitions
\makeatletter
\def\PY@reset{\let\PY@it=\relax \let\PY@bf=\relax%
    \let\PY@ul=\relax \let\PY@tc=\relax%
    \let\PY@bc=\relax \let\PY@ff=\relax}
\def\PY@tok#1{\csname PY@tok@#1\endcsname}
\def\PY@toks#1+{\ifx\relax#1\empty\else%
    \PY@tok{#1}\expandafter\PY@toks\fi}
\def\PY@do#1{\PY@bc{\PY@tc{\PY@ul{%
    \PY@it{\PY@bf{\PY@ff{#1}}}}}}}
\def\PY#1#2{\PY@reset\PY@toks#1+\relax+\PY@do{#2}}

\@namedef{PY@tok@w}{\def\PY@tc##1{\textcolor[rgb]{0.73,0.73,0.73}{##1}}}
\@namedef{PY@tok@c}{\let\PY@it=\textit\def\PY@tc##1{\textcolor[rgb]{0.24,0.48,0.48}{##1}}}
\@namedef{PY@tok@cp}{\def\PY@tc##1{\textcolor[rgb]{0.61,0.40,0.00}{##1}}}
\@namedef{PY@tok@k}{\let\PY@bf=\textbf\def\PY@tc##1{\textcolor[rgb]{0.00,0.50,0.00}{##1}}}
\@namedef{PY@tok@kp}{\def\PY@tc##1{\textcolor[rgb]{0.00,0.50,0.00}{##1}}}
\@namedef{PY@tok@kt}{\def\PY@tc##1{\textcolor[rgb]{0.69,0.00,0.25}{##1}}}
\@namedef{PY@tok@o}{\def\PY@tc##1{\textcolor[rgb]{0.40,0.40,0.40}{##1}}}
\@namedef{PY@tok@ow}{\let\PY@bf=\textbf\def\PY@tc##1{\textcolor[rgb]{0.67,0.13,1.00}{##1}}}
\@namedef{PY@tok@nb}{\def\PY@tc##1{\textcolor[rgb]{0.00,0.50,0.00}{##1}}}
\@namedef{PY@tok@nf}{\def\PY@tc##1{\textcolor[rgb]{0.00,0.00,1.00}{##1}}}
\@namedef{PY@tok@nc}{\let\PY@bf=\textbf\def\PY@tc##1{\textcolor[rgb]{0.00,0.00,1.00}{##1}}}
\@namedef{PY@tok@nn}{\let\PY@bf=\textbf\def\PY@tc##1{\textcolor[rgb]{0.00,0.00,1.00}{##1}}}
\@namedef{PY@tok@ne}{\let\PY@bf=\textbf\def\PY@tc##1{\textcolor[rgb]{0.80,0.25,0.22}{##1}}}
\@namedef{PY@tok@nv}{\def\PY@tc##1{\textcolor[rgb]{0.10,0.09,0.49}{##1}}}
\@namedef{PY@tok@no}{\def\PY@tc##1{\textcolor[rgb]{0.53,0.00,0.00}{##1}}}
\@namedef{PY@tok@nl}{\def\PY@tc##1{\textcolor[rgb]{0.46,0.46,0.00}{##1}}}
\@namedef{PY@tok@ni}{\let\PY@bf=\textbf\def\PY@tc##1{\textcolor[rgb]{0.44,0.44,0.44}{##1}}}
\@namedef{PY@tok@na}{\def\PY@tc##1{\textcolor[rgb]{0.41,0.47,0.13}{##1}}}
\@namedef{PY@tok@nt}{\let\PY@bf=\textbf\def\PY@tc##1{\textcolor[rgb]{0.00,0.50,0.00}{##1}}}
\@namedef{PY@tok@nd}{\def\PY@tc##1{\textcolor[rgb]{0.67,0.13,1.00}{##1}}}
\@namedef{PY@tok@s}{\def\PY@tc##1{\textcolor[rgb]{0.73,0.13,0.13}{##1}}}
\@namedef{PY@tok@sd}{\let\PY@it=\textit\def\PY@tc##1{\textcolor[rgb]{0.73,0.13,0.13}{##1}}}
\@namedef{PY@tok@si}{\let\PY@bf=\textbf\def\PY@tc##1{\textcolor[rgb]{0.64,0.35,0.47}{##1}}}
\@namedef{PY@tok@se}{\let\PY@bf=\textbf\def\PY@tc##1{\textcolor[rgb]{0.67,0.36,0.12}{##1}}}
\@namedef{PY@tok@sr}{\def\PY@tc##1{\textcolor[rgb]{0.64,0.35,0.47}{##1}}}
\@namedef{PY@tok@ss}{\def\PY@tc##1{\textcolor[rgb]{0.10,0.09,0.49}{##1}}}
\@namedef{PY@tok@sx}{\def\PY@tc##1{\textcolor[rgb]{0.00,0.50,0.00}{##1}}}
\@namedef{PY@tok@m}{\def\PY@tc##1{\textcolor[rgb]{0.40,0.40,0.40}{##1}}}
\@namedef{PY@tok@gh}{\let\PY@bf=\textbf\def\PY@tc##1{\textcolor[rgb]{0.00,0.00,0.50}{##1}}}
\@namedef{PY@tok@gu}{\let\PY@bf=\textbf\def\PY@tc##1{\textcolor[rgb]{0.50,0.00,0.50}{##1}}}
\@namedef{PY@tok@gd}{\def\PY@tc##1{\textcolor[rgb]{0.63,0.00,0.00}{##1}}}
\@namedef{PY@tok@gi}{\def\PY@tc##1{\textcolor[rgb]{0.00,0.52,0.00}{##1}}}
\@namedef{PY@tok@gr}{\def\PY@tc##1{\textcolor[rgb]{0.89,0.00,0.00}{##1}}}
\@namedef{PY@tok@ge}{\let\PY@it=\textit}
\@namedef{PY@tok@gs}{\let\PY@bf=\textbf}
\@namedef{PY@tok@gp}{\let\PY@bf=\textbf\def\PY@tc##1{\textcolor[rgb]{0.00,0.00,0.50}{##1}}}
\@namedef{PY@tok@go}{\def\PY@tc##1{\textcolor[rgb]{0.44,0.44,0.44}{##1}}}
\@namedef{PY@tok@gt}{\def\PY@tc##1{\textcolor[rgb]{0.00,0.27,0.87}{##1}}}
\@namedef{PY@tok@err}{\def\PY@bc##1{{\setlength{\fboxsep}{\string -\fboxrule}\fcolorbox[rgb]{1.00,0.00,0.00}{1,1,1}{\strut ##1}}}}
\@namedef{PY@tok@kc}{\let\PY@bf=\textbf\def\PY@tc##1{\textcolor[rgb]{0.00,0.50,0.00}{##1}}}
\@namedef{PY@tok@kd}{\let\PY@bf=\textbf\def\PY@tc##1{\textcolor[rgb]{0.00,0.50,0.00}{##1}}}
\@namedef{PY@tok@kn}{\let\PY@bf=\textbf\def\PY@tc##1{\textcolor[rgb]{0.00,0.50,0.00}{##1}}}
\@namedef{PY@tok@kr}{\let\PY@bf=\textbf\def\PY@tc##1{\textcolor[rgb]{0.00,0.50,0.00}{##1}}}
\@namedef{PY@tok@bp}{\def\PY@tc##1{\textcolor[rgb]{0.00,0.50,0.00}{##1}}}
\@namedef{PY@tok@fm}{\def\PY@tc##1{\textcolor[rgb]{0.00,0.00,1.00}{##1}}}
\@namedef{PY@tok@vc}{\def\PY@tc##1{\textcolor[rgb]{0.10,0.09,0.49}{##1}}}
\@namedef{PY@tok@vg}{\def\PY@tc##1{\textcolor[rgb]{0.10,0.09,0.49}{##1}}}
\@namedef{PY@tok@vi}{\def\PY@tc##1{\textcolor[rgb]{0.10,0.09,0.49}{##1}}}
\@namedef{PY@tok@vm}{\def\PY@tc##1{\textcolor[rgb]{0.10,0.09,0.49}{##1}}}
\@namedef{PY@tok@sa}{\def\PY@tc##1{\textcolor[rgb]{0.73,0.13,0.13}{##1}}}
\@namedef{PY@tok@sb}{\def\PY@tc##1{\textcolor[rgb]{0.73,0.13,0.13}{##1}}}
\@namedef{PY@tok@sc}{\def\PY@tc##1{\textcolor[rgb]{0.73,0.13,0.13}{##1}}}
\@namedef{PY@tok@dl}{\def\PY@tc##1{\textcolor[rgb]{0.73,0.13,0.13}{##1}}}
\@namedef{PY@tok@s2}{\def\PY@tc##1{\textcolor[rgb]{0.73,0.13,0.13}{##1}}}
\@namedef{PY@tok@sh}{\def\PY@tc##1{\textcolor[rgb]{0.73,0.13,0.13}{##1}}}
\@namedef{PY@tok@s1}{\def\PY@tc##1{\textcolor[rgb]{0.73,0.13,0.13}{##1}}}
\@namedef{PY@tok@mb}{\def\PY@tc##1{\textcolor[rgb]{0.40,0.40,0.40}{##1}}}
\@namedef{PY@tok@mf}{\def\PY@tc##1{\textcolor[rgb]{0.40,0.40,0.40}{##1}}}
\@namedef{PY@tok@mh}{\def\PY@tc##1{\textcolor[rgb]{0.40,0.40,0.40}{##1}}}
\@namedef{PY@tok@mi}{\def\PY@tc##1{\textcolor[rgb]{0.40,0.40,0.40}{##1}}}
\@namedef{PY@tok@il}{\def\PY@tc##1{\textcolor[rgb]{0.40,0.40,0.40}{##1}}}
\@namedef{PY@tok@mo}{\def\PY@tc##1{\textcolor[rgb]{0.40,0.40,0.40}{##1}}}
\@namedef{PY@tok@ch}{\let\PY@it=\textit\def\PY@tc##1{\textcolor[rgb]{0.24,0.48,0.48}{##1}}}
\@namedef{PY@tok@cm}{\let\PY@it=\textit\def\PY@tc##1{\textcolor[rgb]{0.24,0.48,0.48}{##1}}}
\@namedef{PY@tok@cpf}{\let\PY@it=\textit\def\PY@tc##1{\textcolor[rgb]{0.24,0.48,0.48}{##1}}}
\@namedef{PY@tok@c1}{\let\PY@it=\textit\def\PY@tc##1{\textcolor[rgb]{0.24,0.48,0.48}{##1}}}
\@namedef{PY@tok@cs}{\let\PY@it=\textit\def\PY@tc##1{\textcolor[rgb]{0.24,0.48,0.48}{##1}}}

\def\PYZbs{\char`\\}
\def\PYZus{\char`\_}
\def\PYZob{\char`\{}
\def\PYZcb{\char`\}}
\def\PYZca{\char`\^}
\def\PYZam{\char`\&}
\def\PYZlt{\char`\<}
\def\PYZgt{\char`\>}
\def\PYZsh{\char`\#}
\def\PYZpc{\char`\%}
\def\PYZdl{\char`\$}
\def\PYZhy{\char`\-}
\def\PYZsq{\char`\'}
\def\PYZdq{\char`\"}
\def\PYZti{\char`\~}
% for compatibility with earlier versions
\def\PYZat{@}
\def\PYZlb{[}
\def\PYZrb{]}
\makeatother


    % For linebreaks inside Verbatim environment from package fancyvrb.
    \makeatletter
        \newbox\Wrappedcontinuationbox
        \newbox\Wrappedvisiblespacebox
        \newcommand*\Wrappedvisiblespace {\textcolor{red}{\textvisiblespace}}
        \newcommand*\Wrappedcontinuationsymbol {\textcolor{red}{\llap{\tiny$\m@th\hookrightarrow$}}}
        \newcommand*\Wrappedcontinuationindent {3ex }
        \newcommand*\Wrappedafterbreak {\kern\Wrappedcontinuationindent\copy\Wrappedcontinuationbox}
        % Take advantage of the already applied Pygments mark-up to insert
        % potential linebreaks for TeX processing.
        %        {, <, #, %, $, ' and ": go to next line.
        %        _, }, ^, &, >, - and ~: stay at end of broken line.
        % Use of \textquotesingle for straight quote.
        \newcommand*\Wrappedbreaksatspecials {%
            \def\PYGZus{\discretionary{\char`\_}{\Wrappedafterbreak}{\char`\_}}%
            \def\PYGZob{\discretionary{}{\Wrappedafterbreak\char`\{}{\char`\{}}%
            \def\PYGZcb{\discretionary{\char`\}}{\Wrappedafterbreak}{\char`\}}}%
            \def\PYGZca{\discretionary{\char`\^}{\Wrappedafterbreak}{\char`\^}}%
            \def\PYGZam{\discretionary{\char`\&}{\Wrappedafterbreak}{\char`\&}}%
            \def\PYGZlt{\discretionary{}{\Wrappedafterbreak\char`\<}{\char`\<}}%
            \def\PYGZgt{\discretionary{\char`\>}{\Wrappedafterbreak}{\char`\>}}%
            \def\PYGZsh{\discretionary{}{\Wrappedafterbreak\char`\#}{\char`\#}}%
            \def\PYGZpc{\discretionary{}{\Wrappedafterbreak\char`\%}{\char`\%}}%
            \def\PYGZdl{\discretionary{}{\Wrappedafterbreak\char`\$}{\char`\$}}%
            \def\PYGZhy{\discretionary{\char`\-}{\Wrappedafterbreak}{\char`\-}}%
            \def\PYGZsq{\discretionary{}{\Wrappedafterbreak\textquotesingle}{\textquotesingle}}%
            \def\PYGZdq{\discretionary{}{\Wrappedafterbreak\char`\"}{\char`\"}}%
            \def\PYGZti{\discretionary{\char`\~}{\Wrappedafterbreak}{\char`\~}}%
        }
        % Some characters . , ; ? ! / are not pygmentized.
        % This macro makes them "active" and they will insert potential linebreaks
        \newcommand*\Wrappedbreaksatpunct {%
            \lccode`\~`\.\lowercase{\def~}{\discretionary{\hbox{\char`\.}}{\Wrappedafterbreak}{\hbox{\char`\.}}}%
            \lccode`\~`\,\lowercase{\def~}{\discretionary{\hbox{\char`\,}}{\Wrappedafterbreak}{\hbox{\char`\,}}}%
            \lccode`\~`\;\lowercase{\def~}{\discretionary{\hbox{\char`\;}}{\Wrappedafterbreak}{\hbox{\char`\;}}}%
            \lccode`\~`\:\lowercase{\def~}{\discretionary{\hbox{\char`\:}}{\Wrappedafterbreak}{\hbox{\char`\:}}}%
            \lccode`\~`\?\lowercase{\def~}{\discretionary{\hbox{\char`\?}}{\Wrappedafterbreak}{\hbox{\char`\?}}}%
            \lccode`\~`\!\lowercase{\def~}{\discretionary{\hbox{\char`\!}}{\Wrappedafterbreak}{\hbox{\char`\!}}}%
            \lccode`\~`\/\lowercase{\def~}{\discretionary{\hbox{\char`\/}}{\Wrappedafterbreak}{\hbox{\char`\/}}}%
            \catcode`\.\active
            \catcode`\,\active
            \catcode`\;\active
            \catcode`\:\active
            \catcode`\?\active
            \catcode`\!\active
            \catcode`\/\active
            \lccode`\~`\~
        }
    \makeatother

    \let\OriginalVerbatim=\Verbatim
    \makeatletter
    \renewcommand{\Verbatim}[1][1]{%
        %\parskip\z@skip
        \sbox\Wrappedcontinuationbox {\Wrappedcontinuationsymbol}%
        \sbox\Wrappedvisiblespacebox {\FV@SetupFont\Wrappedvisiblespace}%
        \def\FancyVerbFormatLine ##1{\hsize\linewidth
            \vtop{\raggedright\hyphenpenalty\z@\exhyphenpenalty\z@
                \doublehyphendemerits\z@\finalhyphendemerits\z@
                \strut ##1\strut}%
        }%
        % If the linebreak is at a space, the latter will be displayed as visible
        % space at end of first line, and a continuation symbol starts next line.
        % Stretch/shrink are however usually zero for typewriter font.
        \def\FV@Space {%
            \nobreak\hskip\z@ plus\fontdimen3\font minus\fontdimen4\font
            \discretionary{\copy\Wrappedvisiblespacebox}{\Wrappedafterbreak}
            {\kern\fontdimen2\font}%
        }%

        % Allow breaks at special characters using \PYG... macros.
        \Wrappedbreaksatspecials
        % Breaks at punctuation characters . , ; ? ! and / need catcode=\active
        \OriginalVerbatim[#1,codes*=\Wrappedbreaksatpunct]%
    }
    \makeatother

    % Exact colors from NB
    \definecolor{incolor}{HTML}{303F9F}
    \definecolor{outcolor}{HTML}{D84315}
    \definecolor{cellborder}{HTML}{CFCFCF}
    \definecolor{cellbackground}{HTML}{F7F7F7}

    % prompt
    \makeatletter
    \newcommand{\boxspacing}{\kern\kvtcb@left@rule\kern\kvtcb@boxsep}
    \makeatother
    \newcommand{\prompt}[4]{
        {\ttfamily\llap{{\color{#2}[#3]:\hspace{3pt}#4}}\vspace{-\baselineskip}}
    }
    

    
    % Prevent overflowing lines due to hard-to-break entities
    \sloppy
    % Setup hyperref package
    \hypersetup{
      breaklinks=true,  % so long urls are correctly broken across lines
      colorlinks=true,
      urlcolor=urlcolor,
      linkcolor=linkcolor,
      citecolor=citecolor,
      }
    % Slightly bigger margins than the latex defaults
    
    \geometry{verbose,tmargin=1in,bmargin=1in,lmargin=1in,rmargin=1in}
    
    

\begin{document}
    
    \maketitle
    
    

    
    \hypertarget{fuzzy-extractors}{%
\section{Fuzzy Extractors}\label{fuzzy-extractors}}

\hypertarget{course-project-by-alexander-krasovskiy}{%
\subsubsection{\texorpdfstring{\emph{Course project by Alexander
Krasovskiy}}{Course project by Alexander Krasovskiy}}\label{course-project-by-alexander-krasovskiy}}

    \hypertarget{ux437ux43cux456ux441ux442}{%
\subsection{Зміст}\label{ux437ux43cux456ux441ux442}}

\begin{enumerate}
\def\labelenumi{\arabic{enumi}.}
\tightlist
\item
  \hyperref[нечіткі-екстрактори]{Вступ}
\item
  \hyperref[постановка-задачі-та-огляд-літератури]{Постановка задачі та огляд літератури}
\item
  \hyperref[необхідні-допоміжні-відомості]{Необхідні допоміжні відомості}
\item
  \hyperref[опис-структури-нечіткого-екстрактора]{Опис структури нечіткого екстрактора}
\item
  \hyperref[аналіз-криптографічної-безпеки-побудованого-нечіткого-екстрактора]{Аналіз криптографічної безпеки побудованого нечіткого екстрактора}
\item
  \hyperref[аналіз-ефективності-побудованої-моделі]{Аналіз ефективності побудованої моделі}
\item
  \hyperref[приклади-використання]{Приклади використання та імітаційні експерименти}
\item
  \hyperref[висновки]{Висновки}
\end{enumerate}

    \hypertarget{ux43dux435ux447ux456ux442ux43aux456-ux435ux43aux441ux442ux440ux430ux43aux442ux43eux440ux438}{%
\subsection{Нечіткі
екстрактори}\label{ux43dux435ux447ux456ux442ux43aux456-ux435ux43aux441ux442ux440ux430ux43aux442ux43eux440ux438}}

    Нечіткі екстрактори --- це метод, який дозволяє використовувати
біометричні дані як вхідні дані для стандартних криптографічних методів
для підвищення їх безпеки. «Нечіткість» у цьому контексті стосується
того факту, що фіксовані значення, необхідні для криптографії, будуть
отримані зі значень, близьких до вихідного ключа, але не ідентичних, без
шкоди для необхідної безпеки.

Нечіткі екстрактори --- це біометричний інструмент, який дозволяє
автентифікувати користувача за допомогою біометричного шаблону,
створеного з біометричних даних користувача як ключа, шляхом вилучення
однорідного випадкового рядка \(R\) з вхідних даних \(w\), з допуском на
шум. Якщо вхідні дані змінюється на \(w'\) але є близькими до \(w\)
відповідно до умов алгоритму, рядок \(R\) буде реконструйовано як на
\(w\), так і на даних \(w'\).

Для досягнення даного результату, під час початкового обчислення \(R\)
процес також повертає допоміжний рядок \(P\), який буде збережено для
відновлення \(R\) і може бути оприлюднений без шкоди для безпеки \(R\).

Безпека процесу також забезпечується, у випадку, коли зловмисник вносить
зміни до значення \(P\).

Після обчислення фіксованого рядка \(R\), Дані можуть бути використані,
наприклад, для узгодження ключів між користувачем і сервером лише на
основі біометричного введення.

    Основні властивості нечітких екстракторів:

\begin{itemize}
\item
  Нечіткий екстрактор є ймовірнісним Монте-Карло алгоритмом, для якого
  виконються умови:

  \begin{itemize}
  \tightlist
  \item
    \(\forall\) неавторизованого користувача \(R^*\), з ймовірністю
    \(1-\epsilon\) дані \(R^*\) відхиляються
  \item
    Для авторизованого користувача \(R\), алгоритм з ймовірністю
    \(1-\epsilon\) (де \(\epsilon\) - ймовірність однобічної помилки
    алгоритму), приймає дані \(R\) та авторизує користувача
  \end{itemize}
\item
  Нечіткий екстрактор є параметричною моделлю. Для довільного нечіткого
  екстрактора, існує набір параметрів \(\theta \in \Theta\), для якого
  ймовірність однобічної помилки є як завгодно малою.
\item
  Нечіткі екстрактори використовуються у комбінації з кодами виправлення
  помилок (англ. Error Correction codes, ECC). Властивості ECC надають
  можливість створення публічного значення \(P\) (check symbols), яке
  може бути оприлюднене без ризиків для безпеки користувача, і
  використане для відновлення ключа за вихідними даними нечіткого
  екстрактора.
\end{itemize}

    \hypertarget{ux43fux43eux441ux442ux430ux43dux43eux432ux43aux430-ux437ux430ux434ux430ux447ux456-ux442ux430-ux43eux433ux43bux44fux434-ux43bux456ux442ux435ux440ux430ux442ux443ux440ux438}{%
\subsection{Постановка задачі та огляд
літератури}\label{ux43fux43eux441ux442ux430ux43dux43eux432ux43aux430-ux437ux430ux434ux430ux447ux456-ux442ux430-ux43eux433ux43bux44fux434-ux43bux456ux442ux435ux440ux430ux442ux443ux440ux438}}

Постановка задачі:

Побудова параметричної моделі нечіткого екстрактору на основі однієї з
моделей face-recognition з відкритим вихідним кодом. Вхідними даними
нечіткого екстрактора є послідовність зображень, та набір параметрів з
простору \(\Theta\). Алгоритм має два режими роботи: створення
перевірочних символів, або ініціалізація алгоритму з існуючими
перевірочними символами і побудова ключа. Результатом роботи алгоритму є
криптографічний ключ, безпечий для використання у інших криптографічних
алгоритмах. За випадково опублікованим ключем, не повино існувати
можливостей отримання біометричних даних користувача.

Огляд літератури:

\begin{itemize}
\tightlist
\item
  Доцільність використання кодів Ріда-Соломона у якості ECC

  \begin{enumerate}
  \def\labelenumi{\arabic{enumi}.}
  \item
    Відповідно до ресурсу:
    https://www.cs.bu.edu/\textasciitilde reyzin/fuzzysurvey.html
    ``Fuzzy Extractors∗ A Brief Survey of Results from 2004 to 2006''
    \textgreater{} The tradeoff between the error tolerance and the
    entropy loss depends on the choice of errorcorrecting code. For
    large alphabets (\(\mathbb{F}\) is a field of size \(\geq n\)),
    \textgreater{} one can use Reed-Solomon codes to get the optimal
    entropy loss of \(2t\log{|\mathbb{F}|}\). \textgreater{} No secure
    sketch construction can have better tradeoff between error tolerance
    and entropy loss than \textgreater{} Construction 1, as searching
    for better secure sketches for the Hamming distance is equivalent to
    searching for \textgreater{} better error-correcting codes. Better
    secure sketches, however, can be achieved if one is willing to
    \textgreater{} slightly weaken the guarantee of correctness (Section
    4).

    (де \(\mathbb{F}\) - алфавіт ECC; t - вага Хемінга векторів, для
    яких вимірюється значення \(\text{loss} = 2t\log{|\mathbb{F}|}\))

    Відповідно до роботи ``Fuzzy Extractors∗ A Brief Survey of Results
    from 2004 to 2006'', даний код конструює безпечну послідовність
    перевірочних символіволів, за якою не існує оптимального алгоритму
    отримання вхідних даних. Також, даний код має оптимальну втрату
    ентропії після додавання перевірочних символів до послідовності:
    \(2t\log{|\mathbb{F}|}\). Результат було враховано під час вибору
    коду ECC та реалізації нечіткого екстрактора.
  \end{enumerate}
\end{itemize}

    \hypertarget{ux43dux435ux43eux431ux445ux456ux434ux43dux456-ux434ux43eux43fux43eux43cux456ux436ux43dux456-ux432ux456ux434ux43eux43cux43eux441ux442ux456}{%
\subsubsection{Необхідні допоміжні
відомості}\label{ux43dux435ux43eux431ux445ux456ux434ux43dux456-ux434ux43eux43fux43eux43cux456ux436ux43dux456-ux432ux456ux434ux43eux43cux43eux441ux442ux456}}

    \hypertarget{ux43cux435ux442ux440ux438ux447ux43dux438ux439-ux43fux440ux43eux441ux442ux456ux440}{%
\subsubsection{Метричний
простір}\label{ux43cux435ux442ux440ux438ux447ux43dux438ux439-ux43fux440ux43eux441ux442ux456ux440}}

Означення \hyperref[література]{(4)}: Метричний простір: Метричний
простір --- це набір \(\mathcal{M}\) із функцією відстані
\(\textit{dis}\) :
\$\mathcal{M}\times\mathcal{M}\longrightarrow \mathbb{R}\^{}+ \$, для
якої виконуються властивості: - нерівність трикутника dis(x, z) ≤
\(\textit{dis}(x, y) + \textit{dis}(y, z)\) - симетрія
\(\textit{dis}(x, y) = \textit{dis}(y, x)\).

Examples: - dis = Hamming distance: \$h(v\_1,v\_2) = \$ sum of one-bits
of \(v_1-v_2\) over \(\mathbb{F}_2\) - dis = set difference: \(\cup\) -
dis = Levenstein distance

    \hypertarget{min-entropy-average-min-entropy-statistical-distance}{%
\subsubsection{Min-Entropy, Average Min-Entropy, Statistical
Distance}\label{min-entropy-average-min-entropy-statistical-distance}}

Означення (min-entropy) (4):

Мінімальною ентропією \(H_\infty(A)\) випадкової величини A називається
значення:

\(~~~~H_\infty(A) = -\log{\underset{a}{\max}}{Pr[A = a]}\)\\
\(~~~~~~~~\) де \(\underset{a}{\max}{Pr[A = a]}\) - передбачуваність А.

Означення (Average Min-Entropy) (4):

Середньою ентропією випадкової величини A за умови B називають значення:

\$\overset{\text{~}}{H}\_\infty(A\textbar B) = \$ todo add descrtiption

    \hypertarget{ux441ux442ux430ux442ux438ux441ux442ux438ux447ux43dux430-ux432ux456ux434ux441ux442ux430ux43dux44c}{%
\subsubsection{Статистична
відстань}\label{ux441ux442ux430ux442ux438ux441ux442ux438ux447ux43dux430-ux432ux456ux434ux441ux442ux430ux43dux44c}}

Означення: Статистична відстань між двома ймовірнісними розподілами
\(A_1\) і \(A_2\):

\(~~~~\text{SD}(A_1,A_2) = \frac{1}{2}\sum\limits_{u}|Pr(A_1~=~u)-Pr(A_2~=~u)|\)

Безпека нечіткого екстрактора зазвичай враховує статистичну\\
\(~~\)відстань між заданим розподілом (ключів екстрактору) і рівномірним
розподілом U.

    \hypertarget{ux437ux430ux445ux438ux449ux435ux43dux456-ux435ux441ux43aux456ux437ux438-ux442ux430-ux43dux435ux447ux456ux442ux43aux456-ux435ux43aux441ux442ux440ux430ux43aux442ux43eux440ux438}{%
\subsubsection{Захищені ескізи та нечіткі
екстрактори}\label{ux437ux430ux445ux438ux449ux435ux43dux456-ux435ux441ux43aux456ux437ux438-ux442ux430-ux43dux435ux447ux456ux442ux43aux456-ux435ux43aux441ux442ux440ux430ux43aux442ux43eux440ux438}}

Захищений ескіз є важливою компонентою нечітких екстракторів. Безпечний
ескіз є схемою,\\
яка приймає в якості вхідної інформації зашумлену інформацію \(\omega\),
наприклад біометричну інформацію,\\
та створює на її основі ескіз \(s\), який є допоміжним рядком.

Означення: Нечітким ескізом називається набір функцій
\((\text{SS},\text{Rec})\), які мають наступні властивості: - Фунція
створення ескізу \(\text{SS}\) на вхідних даних
\(\omega \in \mathcal{M}\) повертає ескіз\\
\(s \in \{0,1\}^*\) - Функція відновлення зашумлених даних
\(\text{Rec}\) для вхідного елементу \(\omega_1 \in \mathcal{M}\) та
ескізу \(s \in \{0,1\}^*\), повертає рядок
\(\omega \Longleftrightarrow \text{dist}(\omega_1,\omega)<t\), де t -
точність нечіткого екстрактора

Захищені схеми ескізу зазвичай використовують коди виправлення помилок,
причому, захищений ескіз відновлює\\
вхідні дані \(\omega_1\) до \(\omega ~~ \Longleftrightarrow\) рядок
\(\omega_1\) є подібним до \(\omega\).\\
Ескіз \(s\) є публічно доступним, оскільки він це розкриває достатньої
інформації про секретні дані \(\omega\).\\
Одним з можливих методів створення безпечного ескізу є БЧХ-коди,
означення яких наведено нижче:

    \hypertarget{ux43aux440ux438ux43fux442ux43eux433ux440ux430ux444ux456ux447ux43dux430-ux441ux442ux456ux439ux43aux456ux441ux442ux44c-ux431ux435ux437ux43fux435ux447ux43dux438ux445-ux435ux441ux43aux456ux437ux456ux432}{%
\subsubsection{Криптографічна стійкість безпечних
ескізів}\label{ux43aux440ux438ux43fux442ux43eux433ux440ux430ux444ux456ux447ux43dux430-ux441ux442ux456ux439ux43aux456ux441ux442ux44c-ux431ux435ux437ux43fux435ux447ux43dux438ux445-ux435ux441ux43aux456ux437ux456ux432}}

Означення: Ескіз \(s\) називається
\((\mathcal{M},m,\overset{\text{~}}{m},t)\) - безпечним,якщо

\begin{itemize}
\tightlist
\item
  \(\forall ~~ W\) - розподілу над метричним простором \(\mathcal{M}\),
  з мінімальною ентропією \(m\),зловмисник має перевагу\\
  обмежену \(2^{-\overset{\text{~}}{m}}\), де
  \(\overset{\text{~}}{m}~\leq~\overset{\text{~}}{H}_{\infty}(W| \text{SS}(W)\)
  для відновлення значення \(W\).
\end{itemize}

    \hypertarget{ux431ux447ux445-ux43aux43eux434}{%
\subsubsection{БЧХ-код}\label{ux431ux447ux445-ux43aux43eux434}}

и є циклічними кодами, які задаються своїм породжуючим поліномом. Для
його знаходження необхідно передусім визначити довжину коду
\({\displaystyle n}\) і мінімальну кодову відстань
\({\displaystyle d\leqslant n}\). Знайти породжуючий поліном можна
наступним чином:

Нехай \({\displaystyle \alpha }\) - примітивний елемент поля
\({\displaystyle GF(q^{m})}\) (тобто
\({\displaystyle \alpha ^{q^{m}-1}=1,\ \alpha ^{i} = 1,\ i<q^{m}-1}\)),
нехай \({\displaystyle \eta =\alpha ^{s}}\) - елемент поля
\({\displaystyle GF(q^{m})}\) порядку \({\displaystyle n}\),
\({\displaystyle s=(q^{m}-1)/n}\). Тоді нормований поліном
\({\displaystyle g(x)}\) мінімальної степені над полем
\({\displaystyle GF(q)}\), коренями якого є \({\displaystyle d-1}\)
послідовних степеней
\({\displaystyle \eta ^{l_{0}},\eta ^{l_{0}+1},\ldots ,\eta ^{l_{0}+d-2}}\)
елемента \({\displaystyle \eta }\) для деякого цілого
\({\displaystyle l_{0}}\) (в тому числі 0 і 1), є породжуючим поліномом
БЧХ-коду над полем \({\displaystyle GF(q)}\) з довжиною
\({\displaystyle n}\) і мінімальною відстанню
\({\displaystyle d_{0}\geqslant d}\).

Пояснимо, чому у отриманого коду будуть саме такі характеристики
(довжина коду \({\displaystyle n}\), мінімальна відстань
\({\displaystyle d_{0}}\)). Справді, як показано у
\hyperref[література]{(8)}, довжина БЧХ-коду дорівнює порядку елемента
\({\displaystyle \eta }\), якщо \({\displaystyle d>2}\), і дорівнює
порядку елемента \({\displaystyle \eta ^{l_{0}}}\), якщо
\({\displaystyle d=2}\). Оскільки випадок \({\displaystyle d=2}\) нас не
цікавить (такий код може виявляти, але не може виправляти помилки),
довжина коду буде дорівнювати порядку елемента
\({\displaystyle \eta }\), тобто дорівнюватиме \({\displaystyle n}\).
Мінімальна відстань \({\displaystyle d_{0}}\) може бути більшою за
\({\displaystyle d}\), коли коренями мінімальних функцій
\hyperref[література]{(8)} від елементів
\({\displaystyle \eta ^{l_{0}},\eta ^{l_{0}+1},\ldots ,\eta ^{l_{0}+d-2}}\)
будуть елементи, які доповнюють послідовність, тобто елементи
\({\displaystyle \eta ^{l_{0}+d-1},\eta ^{l_{0}+d},\ldots ,\eta ^{l_{0}+d_{0}-2}}\).

Кількість перевірочних символів \({\displaystyle r}\) дорівнює степеню
\({\displaystyle g(x)}\), кількість інформаційних символів
\({\displaystyle k=n-r}\), величина \({\displaystyle d}\) називається
конструктивною відстанню БЧХ-коду. Якщо \({\displaystyle n=q^{m}-1}\),
то код називається примітивним, інакше, непримітивним.

Так само, як і для циклічного коду, кодовий поліном
\({\displaystyle c(x)}\) може бути отриманий з інформаційного поліному
\({\displaystyle m(x)}\) степені не більше \({\displaystyle k-1}\),
шляхом перемноження \({\displaystyle m(x)}\) і \({\displaystyle g(x)}\).

    \hypertarget{ux43aux43eux434ux438-ux440ux456ux434ux430-ux441ux43eux43bux43eux43cux43eux43dux430-7}{%
\subsubsection{\texorpdfstring{Коди Ріда-Соломона
\hyperref[література]{(7)}}{Коди Ріда-Соломона }}\label{ux43aux43eux434ux438-ux440ux456ux434ux430-ux441ux43eux43bux43eux43cux43eux43dux430-7}}

Коди Ріда - Соломона є важливим окремим випадком БЧХ-коду, корені
породжуючого многочлена якого лежать у тому ж полі, над яким будується
код. Нехай \(\alpha\) --- елемент поля \(\textstyle GF(q)\), що має
порядок \(\textstyle n\). Якщо \(\alpha\)~--- примітивний елемент, його
порядок дорівнює \(q-1\), тобто
\$\alpha\textsuperscript{\{q-1\}=1,\quad \alpha}i \neq 1,
0\textless i\textless q-1 \$. Тоді нормований поліном \(g(x)\)
мінімального ступеня над полем \(\textstyle GF(q)\), коренем якого є
\(𝑑−1\) послідовних ступенів
\(\alpha^{l_0}, \alpha^{l_0+1},...,\alpha^{l_0+d-2}\) елемента
\(\alpha\), є породжуючим многочленом коду Ріда --- Соломона над полем
\(\textstyle GF(q)\):

\(g(x) = (x - \alpha^{l_0})(x - \alpha^{l_0+1})\dots(x - \alpha^{l_0+d-2}),\)

де \(l_0\) - деяке ціле число (у тому числі 0 і 1), зазвичай обирається
\(𝑙_0 = 1\). Ступінь многочлена \$ g (x) \$ дорівнює \$ d-1 \$. Довжина
отриманого коду \(n\), мінімальна відстань \(d\)(є мінімальною з усіх
відстаней Хеммінга всіх пар кодових слів, див. Лінійний код). Код
містить \(r=d-1=\deg (g(x))\) перевірочних символів, де \(\deg()\)
позначає ступінь полінома; число інформаційних символів: \$k = n - r = n
- d + 1 \$. Таким чином, \(\textstyle d = n - k + 1\) і код Ріда~---
Соломона є кодом, що має максимальну кодову відстань (є оптимальним у
сенсі границі Сінглтона). Кодовий поліном \(c(x)\) може бути отриманий з
інформаційного полінома \(m(x)\),\(\deg m(x) \leqslant k-1\) шляхом
перемноження \(m(x)\) і \(g(x)\):

\$ c (x) = m (x) g (x) \$.

    \hypertarget{ux43dux435ux447ux456ux442ux43aux456-ux435ux43aux441ux442ux440ux430ux43aux442ux43eux440ux438}{%
\subsubsection{Нечіткі
екстрактори}\label{ux43dux435ux447ux456ux442ux43aux456-ux435ux43aux441ux442ux440ux430ux43aux442ux43eux440ux438}}

Означення: Нечітким екстрактором називаються два ймовірнісні алгоритми
\(\text{Gen}, \text{Rep}\) з наступними властивостями:

\begin{itemize}
\tightlist
\item
  Генеруюча функція \(Gen(x) \rightarrow (R,P)\), на вхідних даних
  \(x \in \mathcal{M}\), повертає рядок \(R \in \{0,1\}^l\) і допоміжні
  дані \(P \in \{0,1\}^*\)
\item
  Функція відтворення вхідних даних \(Rep\), на вхідному елементі
  \(\omega_1 \in \mathcal{M}\), і допоміжних даних \(P \in \{0,1\}^*\),
  повертає рядок \(R\) такий що:\\
  \(\text{Rep}(\omega_1,P) \rightarrow R~~\text{якщо}~  dis(x,\omega_1)\leq t\)
\end{itemize}

    \hypertarget{ux43aux440ux438ux43fux442ux43eux433ux440ux430ux444ux456ux447ux43dux430-ux441ux442ux456ux439ux43aux456ux441ux442ux44c-ux43dux435ux447ux456ux442ux43aux438ux445-ux435ux43aux441ux442ux440ux430ux43aux442ux43eux440ux456ux432}{%
\subsubsection{Криптографічна стійкість нечітких
екстракторів}\label{ux43aux440ux438ux43fux442ux43eux433ux440ux430ux444ux456ux447ux43dux430-ux441ux442ux456ux439ux43aux456ux441ux442ux44c-ux43dux435ux447ux456ux442ux43aux438ux445-ux435ux43aux441ux442ux440ux430ux43aux442ux43eux440ux456ux432}}

Означення: Нечіткий екстрактор називається
\((\mathcal{M},m,l,t,\epsilon)\)-безпечним, якщо

\begin{itemize}
\tightlist
\item
  \(\forall ~~ W\) - розподілу вхідних даних над метричним простором
  \(\mathcal{M}\), з\\
  мінімальною ентропією \(m\), вихідні дані (за виключенням допоміжних),
  розподілені\\
  згідно з розподілом \(R\), і виконуються умови:

  \begin{itemize}
  \tightlist
  \item
    Статистична відстань між \(R\) та \(U_l\) - рівномірним\\
    розподілом розмірності \(l\), є незначною навіть якщо\\
    допоміжні дані \(P\) є доступними зловмиснику. Тобто,
    \(\textbf{SD}((R,P),(U_l,P)) \leq \epsilon\), де \(\epsilon\) є
    малим числом.
  \end{itemize}
\end{itemize}

    \hypertarget{ux43cux43eux434ux435ux43bux44c-face-recognition}{%
\subsubsection{Модель Face
recognition}\label{ux43cux43eux434ux435ux43bux44c-face-recognition}}

Source-код моделі: \url{https://github.com/ageitgey/face_recognition}

\begin{enumerate}
\def\labelenumi{\arabic{enumi}.}
\tightlist
\item
  Модель Face recognition є бібліотекою з відкритим кодом, ліцензованою
  згідно з \(\text{MIT License}\)
\item
  Модель побудована на основі deep face-recognition model, реалізованій
  у бібліотеці \url{http://dlib.net/}
\item
  Більш детально розглянуто метод
  \texttt{face\_recognition.api.face\_distance}, для отримання формату
  представлення face-вектору \(v\):
  \(v \in \mathbb{F}_2^{128}; \space d(v_1,v_2) = ||v_2-v_1||, \space \mathbb{F}_2 = \{0,1\}\)
\item
  Відповідно до документації бібліотеки \texttt{dlib} та методу
  \texttt{face\_recognition.api.face\_distance}, для отримання
  статистики, яка визначає подібність обличь за їх face-вектором, автор
  використовує базову Евклідову метрику у просторі розмірності \(128\).
\end{enumerate}

    \hypertarget{ux43eux43fux438ux441-ux441ux435ux440ux435ux434ux43eux432ux438ux449ux430-ux440ux43eux437ux440ux43eux431ux43aux438}{%
\subsection{Опис середовища
розробки}\label{ux43eux43fux438ux441-ux441ux435ux440ux435ux434ux43eux432ux438ux449ux430-ux440ux43eux437ux440ux43eux431ux43aux438}}

Environment info: - Platform: \texttt{Docker} - Container image:
\texttt{python:3.10} - SSH port: \texttt{2222} - IDE:
\texttt{PyCharm\ 2023.1\ RC\ 2} - Base: \texttt{Windows\ 11\ (WSL2)}

dependencies (direct): - itertools - numpy\textasciitilde=1.24.2 -
opencv-python\textasciitilde=4.7.0.72 - face\_recognition -
Pillow\textasciitilde=9.4.0 - PIL - hashlib

dependencies (transitive, manually installed) - ffmpeg - CMake - dlib -
libsm6 - libxext6 - libsasl2-dev - python-dev - libldap2-dev -
libssl-dev - openssh-server

\hypertarget{ux43fux435ux440ux435ux43bux456ux43a-ux43aux43bux430ux441ux456ux432-ux442ux430-ux43cux43eux434ux443ux43bux456ux432-ux43fux440ux43eux435ux43aux442ux443}{%
\subsection{Перелік класів та модулів
проекту}\label{ux43fux435ux440ux435ux43bux456ux43a-ux43aux43bux430ux441ux456ux432-ux442ux430-ux43cux43eux434ux443ux43bux456ux432-ux43fux440ux43eux435ux43aux442ux443}}

python Modules: - extractors.py: Реалізація нечіткого екстрактора -
cache.py: Кешування обєктів, пришвидшення повторного виконання тестів -
logger.py - main.py: - testing.py: додаткові класи для тестування
нечіткого екстрактора - test\_face\_recognition.py: unit-тести окремих
класів та модулів

python Classes: - Cache - кешування обєктів - FrameIterator - ітерування
зображень у відео-файлі - FaceVectorExtractor - допоміжні функції для
отримання face-вектору за зображенням - FuzzyExtractorFaceRecognition -
клас нечіткого екстрактора. Основна логіка екстрактора зосереджена у
даному класі - LogFormatter - форматування виводу (stdin/log file) -
TestFaceVectorExtractor - основний тестовий клас - TestCases -
допоміжний клас для проведення тестування

    \hypertarget{ux43eux43fux438ux441-ux441ux442ux440ux443ux43aux442ux443ux440ux438-ux43dux435ux447ux456ux442ux43aux43eux433ux43e-ux435ux43aux441ux442ux440ux430ux43aux442ux43eux440ux430}{%
\subsection{Опис структури нечіткого
екстрактора}\label{ux43eux43fux438ux441-ux441ux442ux440ux443ux43aux442ux443ux440ux438-ux43dux435ux447ux456ux442ux43aux43eux433ux43e-ux435ux43aux441ux442ux440ux430ux43aux442ux43eux440ux430}}

\begin{itemize}
\tightlist
\item
  Input data preprocessing:

  \begin{enumerate}
  \def\labelenumi{\arabic{enumi}.}
  \tightlist
  \item
    Перетворення вхідних даних до вигляду:
    \texttt{list{[}PIL.Image.Image{]}}
  \item
    Перетворення: \texttt{PIL.Image.Image}
    \(\overset{convert: RGB}{\longrightarrow}\) \texttt{np.ndarray}
  \item
    Визначення face-вектору для кожного з зображень. Зображення для яких
    не було знайдено жодного face-вектору, або було знайдено декілька
    face-векторів, далі не обробляються.
  \item
    Повернення послідовності face-векторів у вигляді
    \texttt{np.ndarray{[}np.ndarray{]}}
  \end{enumerate}
\end{itemize}

    \begin{itemize}
\tightlist
\item
  Face vector preprocessing (Input data: \(V\) -
  \texttt{np.ndarray{[}np.ndarray{]}}):

  \begin{enumerate}
  \def\labelenumi{\arabic{enumi}.}
  \tightlist
  \item
    Обчислення
    \(S = \{(\#|v_1-v_2|, v_1, v_2) \space | \space v_1, v_2 \in V, v_1\neq v_2\}\)
    (1)
  \item
    Обчислення викидів вибірки \(S\) (\(S^*\)) за значеннями першої
    координати:

    \begin{itemize}
    \item
      Нехай \(\mu\) - середнє значення вибірки за першою координатою:
      \(\mu = \#\{v[0]\space,\space v \in S\}\),

      \(\sigma\) - стандартне відхилення вибірки за першою координатою:
      \(\sigma = \sqrt{\#\{(v[0]-\mu)^2\space,\space v \in S\}}\)

      \(\sigma_0\) - максимальне стандартне відхилення, параметр
    \item
      \(v \in S\) будемо вважати викидом, якщо для \(v\) виконується:
      \(|\mu-v[0]| > \sigma_0\)
    \end{itemize}
  \item
    Створення вибірки \(S_1\) за елементами вибірки
    \(H = S\setminus S^*\):

    \begin{itemize}
    \tightlist
    \item
      \(S_1 = \{(|\{s| v=s[1] \lor v=s[2], \space s \in H\}|, v) \space|\space v \in \{s[1],s[2] \space|\space s \in H\}\}\)
    \end{itemize}
  \item
    Знаходження викидів вибірки \(S_1\): \(S_1^*\)
  \item
    Повернення \(\{v[1]\space|\space v \in S_1 \setminus S_1^*\}\) (тип
    даних: \texttt{np.ndarray{[}np.ndarray{]}})
  \end{enumerate}

  \begin{itemize}
  \tightlist
  \item
    Передумови:

    \begin{enumerate}
    \def\labelenumi{\arabic{enumi}.}
    \tightlist
    \item
      Для обчислення методу (1), необхідною передумовою є взаємна
      незалежність координат face-вектору. Оскільки face-vector належить
      простору \((-1,1)^{128} \subset \mathbb{Q}^{128}\), з метрикою
      Евкліда, можна зробити висновок, що кожна координата має однаковий
      вплив на ідентифікацію обличчя. Також, відповідно до проведених
      тестів (dlib) не було знайдено кореляції між координатами
      face-вектору, що дозволяє зробити припущення про їх незалежність.
    \end{enumerate}
  \item
    Зауваження

    \begin{enumerate}
    \def\labelenumi{\arabic{enumi}.}
    \setcounter{enumi}{1}
    \tightlist
    \item
      Для більш точної побудови ключа, параметр \(\sigma_0\) може бути
      зменшений (значення за замовчуванням: 0.7). Дана змінна є одним з
      параметрів моделі нечіткого екстрактора.
    \end{enumerate}
  \end{itemize}
\end{itemize}

    \begin{itemize}
\tightlist
\item
  Primary hash (первинний хеш): (Input data:
  \texttt{np.ndarray{[}np.ndarray{]}})

  \begin{enumerate}
  \def\labelenumi{\arabic{enumi}.}
  \item
    Нехай простір \(M \subset (-1,1)^{128} \subset \mathbb{Q}^{128}\)
    розбито на множини:

    \begin{enumerate}
    \def\labelenumii{(\arabic{enumii})}
    \item
      \(M_{i_1,i_2,...,i_{128}}\):
      \(\space \underset{(i_1,..,i_{128}) \in I}{\large{\cup}}  {\small{M}_{i_1,i_2,...,i_{128}}} = M\)
      (\(I\) - множина індексів) за правилом:
    \item
      \(M_{i_1,i_2,...,i_{128}}\) =
      \((i_1d,(i_1+1)d)\times(i_2d,(i_2+1)d)\times...\times(i_{128}d,(i_{128}-1)d)\)
      - (гіперкуб зі стороною \(d\); \(d\) - параметр моделі нечіткого
      екстрактора).
    \end{enumerate}

    Очевидно, для покриття простору (2), виконується (1), якщо M не
    містить граней гіперкубів \(M_{i_1,i_2,...,i_{128}}\). У
    реалізованому екстракторі, дане твердження є передумовою прийняття
    вхідних даних:

    \begin{quote}
    Якщо принаймні одна координата face-вектору знаходиться на перетині
    граней двох або більшої кількості гіперкубів, даний face-вектор не
    може бути оброблені нечітким екстрактором, тому вхідні дані будуть
    відхилені. (На практиці, ймовірність такої події дуже мала, під час
    тестування вхідні дані не були відхилені жодного разу).
    \end{quote}

    Параметр \(d \in (0,1)\), є параметром нечіткого екстрактора, який
    впливає на рівень, на якому подібні face-вектори будуть вважатися
    ідентичними.
  \item
    Нехай \(\text{std_max},\alpha\) - параметри нечіткого екстрактора.
    Нехай \(\overline{v},\overset{\text{~}}{v}\)
  \end{enumerate}

  \begin{itemize}
  \tightlist
  \item
    вектори розмірності 128, \(\overline{v}\) - містить середні
    значення, \(\overset{\text{~}}{v}\) - стандартні відхилення
    face-векторів по кожній координаті.

    \begin{itemize}
    \tightlist
    \item
      Якщо для вхідних даних виконується:
      \(\#\{\overset{\text{~}}{v}[i]>\text{std_max}\space|\space i = \overline{\small{1,\dim{\overset{\text{~}}{v}}}}\} \gt  \alpha \dim(\overset{\text{~}}{v})\),
      модель нечіткого екстрактора відхилить вхідні дані, оскільки для
      обраних параметрів стандартне відхилення деякої множини координат
      перевищує задане максимально допустиме значення.
    \end{itemize}
  \end{itemize}

  \begin{enumerate}
  \def\labelenumi{\arabic{enumi}.}
  \setcounter{enumi}{2}
  \tightlist
  \item
    До даних (\(V\)) застосовано наступне перетворення:

    \begin{itemize}
    \item
      Нехай
      \(v \in V, v \in M_{I_0}, I_0 \in I, I_0 = (i_1,i_2,...,i_{128})\),
      \(f: V \rightarrow M\)
    \item
      Побудуємо функцію \(f\):

      \(f(v): v \longrightarrow (i_1+\frac{d}{2},i_2+\frac{d}{2},..,i_{128}+\frac{d}{2})\)

      Таким чином, кожен face-вектор, який належить гіперкубу
      \(M_{I_0}\), буде відображено у центр \(M_{I_0}\)
    \end{itemize}
  \item
    Алгоритм повертає представлення значення
    \(f(\space\overline{v}\space)\) у вигляді послідовності байтів
    \texttt{bytes}
  \end{enumerate}

  \begin{itemize}
  \tightlist
  \item
    Зауваження:

    \begin{itemize}
    \tightlist
    \item
      Покриття простору підмножинами у вигляді гіперкубів було обрано на
      перевагу розбиттю простору на сфери, оскільки реалізація даного
      розбиття є досить простою для просторів розмірності більшої за 3,
      а обчислювана складність задачі є найменшою серед усіх можливих
      розбиттів: \(O(1)\). Задача ефективного покриття простору
      розмірності 128 на сфери, на даний момент не була розв'язана. Тому
      не існує ефективного алгоритму за яким було б можливо побудувати
      таке розбиття.
    \item
      За рахунок розбиття (1), вектори \(v_1,v_2 \in V\), однаково
      віддалені від центру \(M_{I_0}\): \(c_0\), такі що
      \(|v_1-c_0|\leq d,|v_2-c_0|\leq d\)), можуть бути відображені у
      центри різних елементів розбиття, що вважатиметься помилкою
      обчислень. Для виправлення даної помилки реалізовано метод, опис
      якого наведено нижче.
    \item
      Primary hash не є стійким до взяття прообразу
    \item
      Primary hash не є стійким до колізій
    \item
      Primary hash не є стійким до взяття другого прообразу
    \end{itemize}
  \end{itemize}
\end{itemize}

    \begin{itemize}
\tightlist
\item
  Primary hash error correction, Input data:
  \texttt{np.ndarray{[}np.ndarray{]}}

  \begin{enumerate}
  \def\labelenumi{\arabic{enumi}.}
  \tightlist
  \item
    Виконання \(N\) незалежних тестів \texttt{hash\_primary}, побудова
    вибірки \(H\) яка містить \(N\) хеш значень.
  \item
    Обчислення ймовірностей \(p_h\) появи значень
    \(h \in \text{set}(H)\)
  \item
    Нехай \(\text{max_unique_hashes}\) - параметр моделі нечіткого
    екстрактора, який визначає максимально можливу кількість унікальних
    хеш-значень, допустиму у вибірці \(H\)
  \item
    Відповідно, якщо вхідні дані задовільняють нерівність:
    \(|set(H)| \gt \text{max_unique_hashes}\): алгоритм відхиляє вхідну
    послідовність face-векторів. Дана нерівність справджується, якщо
    після обробки face-векторів та аналізу вибірки на предмет викидів,
    оцінка хеш-значення є недостатньо ефективною. (за замовчуванням,
    \(\text{max_unique_hashes} = 1000\))
  \item
    Нехай \(\text{p_a_min} \in (0,1)\) - параметр моделі нечіткого
    екстрактора
  \item
    Якщо \(\exists h \in H: p_h>\text{p_a_min}\), результатом роботи
    алгоритму є значення \(h\)
  \item
    Інакше, знайдемо значення \(h,l \in H\), які мають найбільшу
    ймовірність появи \(p_h,p_l\).
  \item
    Визначимо значення ентропії Шенона (над алфавітом, який містить всі
    можливі байти), послідовностей \(h,l \in H\): \(e_h, e_l\)
  \item
    Результатом роботи алгоритму є вектор з найбільшим значенням
    ентропії. Якщо \(e_h = e_l\), повертається послідовність випадкова
    послідовність (\(l\) або \(h\)).
  \end{enumerate}
\end{itemize}

    \begin{itemize}
\tightlist
\item
  Secondary hash

  \begin{itemize}
  \tightlist
  \item
    Перетворення вихідних даних алгоритму primary hash та додаткових
    перевірочних символів, отримання безпечного криптографічного ключа.
  \end{itemize}

  \begin{enumerate}
  \def\labelenumi{\arabic{enumi}.}
  \tightlist
  \item
    Нехай \(\text{check_symbols_count}\) - параметр нечіткого
    екстрактора, який відповідає за кількість перевірочних символів, які
    будуть згенеровані для вихідних даних \(\text{hash_primary}\)
    використовуючи код Ріда-Соломона. (за замовчуванням,
    \(\text{check_symbols_count} = 32\), тобто максимально дозволена
    похибка обчислення \(\text{hash_primary}\): 16 координат
    face-вектору)
  \item
    Вхідними даними алгоритму \(\text{hash_secondary}\) є первинне
    \(\text{hash_primary}\), та послідовність перевірочних символів
    \(\text{check_symbols}\).
  \item
    Обчислюється рядок \(\text{hash_primary}+\text{check_symbols}\),
    помилки рядка виправляються за допомогою стандартного коду
    Ріда-Соломона. Якщо кількість помилок перевищує
    \(\frac{\text{check_symbols_count}}{2}\), код Ріда-Соломона не
    декодує вхідні дані, тому під час виникнення помилки дані будуть
    відхилені (користувач не пройшов аутентифікацію).
  \item
    Результатом роботи методу hash\_secondary, є послідовність байтів,
    яка може бути безпечно використана для
    ініціалізації/шифрування/створення цифрового підпису користувача та
    інших криптографічних алгоритмів. Метод \texttt{hash\_secondary}
    додатково містить аргумент \texttt{salt} (сіль, яка буде використана
    під час створення безпечного хеш-значення SHA256), збереження
    послідовностей \texttt{salt} та \texttt{check\_symbols} передбачено
    у окремій базі даних без необхідності додаткового шифрування.
  \end{enumerate}

  \begin{itemize}
  \tightlist
  \item
    Властивості:

    \begin{itemize}
    \tightlist
    \item
      \texttt{hash\_secondary} є стійким до колізій
    \item
      \texttt{hash\_secondary} є стійким до взяття прообразу
    \item
      \texttt{hash\_secondary} є стійким до взяття другого прообразу
    \end{itemize}
  \item
    Зауваження:

    \begin{itemize}
    \tightlist
    \item
      Алгоритм \texttt{hash\_secondary} генерує криптографічно безпечний
      ключ, що дозволяє уникнути потенційних вразливостей, пов'язаних з
      отриманням зловмисником доступу до ключа: за значенням ключа,
      згенерованого \texttt{hash\_secondary}, не існує ефективного
      алгоритму знаходження face-вектору користувача.
    \item
      Використання солі (\texttt{salt}) під час хешування
      \(\text{hash_primary}\) рекомендується, оскільки це дозволяє
      запобігти вразливостям типу \(\text{Rainbow table attack}\)
    \end{itemize}
  \end{itemize}
\end{itemize}

    \hypertarget{ux430ux43dux430ux43bux456ux437-ux43aux440ux438ux43fux442ux43eux433ux440ux430ux444ux456ux447ux43dux43eux457-ux431ux435ux437ux43fux435ux43aux438-ux43fux43eux431ux443ux434ux43eux432ux430ux43dux43eux433ux43e-ux43dux435ux447ux456ux442ux43aux43eux433ux43e-ux435ux43aux441ux442ux440ux430ux43aux442ux43eux440ux430.}{%
\subsubsection{Аналіз криптографічної безпеки побудованого нечіткого
екстрактора.}\label{ux430ux43dux430ux43bux456ux437-ux43aux440ux438ux43fux442ux43eux433ux440ux430ux444ux456ux447ux43dux43eux457-ux431ux435ux437ux43fux435ux43aux438-ux43fux43eux431ux443ux434ux43eux432ux430ux43dux43eux433ux43e-ux43dux435ux447ux456ux442ux43aux43eux433ux43e-ux435ux43aux441ux442ux440ux430ux43aux442ux43eux440ux430.}}

Надалі будемо вважати, що значення face-векторів, згенерованих
бібліотекою \texttt{dlib}, мають рівномірний розподіл \(U(0,1)^{128}\)

\begin{enumerate}
\def\labelenumi{\arabic{enumi}.}
\tightlist
\item
  Аналіз статистичної відстані між розподілом ключів, які генеруються
  нечітким екстрактором, та рівномірним розподілом \(U\):

  \begin{itemize}
  \tightlist
  \item
    Ключові моменти:

    \begin{enumerate}
    \def\labelenumii{\arabic{enumii}.}
    \tightlist
    \item
      Відображення простору face-векторів у методі
      \texttt{hash\_primary}
    \item
      Виправлення помилок у вихідному рядку за допомогою кодів
      Ріда-Соломона.
    \item
      Хешування вихідних даних (SHA256+salt)
    \end{enumerate}
  \item
    Аналіз відображення, реалізованого у методі \texttt{hash\_primary}:

    \begin{enumerate}
    \def\labelenumii{\arabic{enumii}.}
    \tightlist
    \item
      Вхідними даними є вектори розмірності 128 з рівномірного росподілу
      \(U(0,1)^{128}\)
    \item
      Кожен вектор відображається у центр деякого гіперкубу з довжиною
      ребра \(d\). Відображення переводить однакову кількість векторів
      до кожного з гіперкубів розбиття, тому у вихідних даних
      зберігається рівномірний розподіл даних. Статистична відстань
      \(\text{SD}(A_1,A_2) = \frac{1}{2}\sum\limits_{u}|Pr(A_1~=~u)-Pr(A_2~=~u)|\),
      рівна \(\frac{1}{2}\frac{1}{d^{128}}|0~-~d^{128}|~=~\frac{1}{2}\).
      Отже, розподіл вихідних даних не дорівнює \(U(0,1)^{128}\), але
      низьке значення статистичної відстані вказує на значну подібність
      даних розподілів. Тому, відповідно до\\
      аналізу статистичної відстані, відображення побудовано коректно
      (дискретизація \(U(0,1)^{128}\)).
    \end{enumerate}
  \item
    Аналіз перетворення коду Ріда Соломона

    \begin{enumerate}
    \def\labelenumii{\arabic{enumii}.}
    \tightlist
    \item
      Вхідними даними є значення \texttt{hash\_primary} з рівномірного
      росподілу, отриманого дискретизацією \(U(0,1)^{128}\), та
      послідовність перевірочних символів \(s\). Оскільки значення для
      яких були згенеровані перевірочні символи є нормально
      розподіленими, враховуючи властивості БЧХ-кодів, будемо вважати що
      послідовність \(s\) також є рівномірно розподіленою величиною.
    \item
      Тоді для кожної пари \((x,s)\), код Ріда-Соломона виправляє
      фіксовану кількість помилок, або відхиляє вхідні дані. Розглядаючи
      простір вхідних рядків з метрикою \(L\) - відстань Левенштейна,
      приходимо до висновку, що
      \(\forall x \in \mathcal{M}, \forall s \in \{0,1\}^* - \text{фіксованого}\):
      Код Ріда-Соломона перетворює однакову кількість рядків \(x\) у
      рядок, якому відповідає послідовність перевірочних символів
      (рядки, у яких значення метрики Левенштейна менше ніж задане
      фіксоване значення \(m\), причому кількість перевірочних символів
      \(|s| \geq 2m\)). Тому при перетворенні вхідних даних за допомогою
      кода Ріда-Соломона, рівномірна розподіленість зберігається.
    \end{enumerate}
  \item
    Хешування за допомогою \(\text{SHA256} + \text{salt}\).

    \begin{enumerate}
    \def\labelenumii{\arabic{enumii}.}
    \tightlist
    \item
      Вважатимемо, що послідовність \(\text{salt}\) є рівномірно
      розподіленою величиною.
    \item
      За означенням, \(\text{SHA256}\) є криптографічно безпечною хеш
      функцією, з чого випливає наступна властивість:\\
      SHA256 перетворює вхідні рядки, які є рівномірно розподіленими, у
      послідовність хеш-символів, яка також є рівномірно розподіленою
      величиною.\\
      Тому, перетворення хешування (\(\text{SHA256} + \text{salt}\)),
      зберігає рівномірний розподіл вхідних даних, за у мови що
      послідовність \(\text{salt}\) є рівномірно розподіленою величиною.
    \end{enumerate}
  \item
    \(\Longrightarrow\) Отже, за результатом аналізу статистичної
    відстані розподілу ключів, можемо зробити висновок про відсутність
    вразливостей пов'язаних з нерівномірною ймовірністю отримання деяких
    ключів. Тому нечіткий екстрактор (з точки зору розподілу ключів), є
    криптографічно безпечним.
  \end{itemize}
\item
  Аналіз на криптографічну безпеку з точки зору хеш-функцій.

  \begin{itemize}
  \tightlist
  \item
    Вимоги до будь-якого криптографічно стійкого нечіткого екстрактора,
    включають деякі властивості\\
    криптографічно безпечних хеш-функцій:

    \begin{enumerate}
    \def\labelenumii{\arabic{enumii}.}
    \tightlist
    \item
      Стійкість до взяття першого прообразу
    \item
      Стійкість до колізій
    \item
      Стійкість до взяття другого прообразу
    \end{enumerate}
  \item
    Стійкість до взяття першого прообразу:\\
    Дана властивість випливає з криптографічної безпеки функції
    \(\text{SHA256}\), оскільки вхідні дані\\
    даного алгоритму є рівномірно розподіленими.
  \item
    Стійкість до колізій:\\
    \(~~\) Для алгоритму нечіткого екстрактора можна стверджувати про
    стійкість до колізій елементів, які не є\\
    \(~~\) подібними у метричному просторі \(\mathcal{M}\) (подібними є
    елементи, для яких значення метрики\\
    \(~~\) простору \(\mathcal{M}\) менше заданого порогового значення
    \(d\), яке є точністю нечіткого екстрактора)
  \item
    Стійкість до взяття другого прообразу:\\
    Алгоритм є стійким до взяття другого прообразу \(\forall\) вхідних
    рядків \(x,x_1~\in~\mathcal{M}\), які не є подібними. Тобто, не
    існує ефективного алгоритму отримання рядку \(x_1 \not \simeq x\),
    для якого значення ключа є рівне значенню ключа вхідних даних \(x\)
  \end{itemize}
\end{enumerate}

    \begin{enumerate}
\def\labelenumi{\arabic{enumi}.}
\setcounter{enumi}{2}
\tightlist
\item
  Аналіз безпеки ескізу нечіткого екстрактора:

  \begin{itemize}
  \tightlist
  \item
    Нехай задано Код Ріда-Соломона з параметрами:

    \begin{itemize}
    \tightlist
    \item
      \(n\) - кількість символів рядка, що кодується, причому
      \(n = 2^m-1\)
    \item
      \(t\) - кількість перевірочних символів
    \end{itemize}
  \item
    Нехай \(\mathscr{U} = GF(2^m)^*\) - скінченне поле порядку \(m\)
  \item
    Нехай \(SDif(\mathscr{U})\) - метричний простір з метрикою, яка
    повертає кількість елементів у симетричній різниці множин
    \(s,s_1~\in~\mathscr{U}\) на всіх підмножинах множини
    \(\mathscr{U}\)
  \item
    Тоді Код Ріда-Соломона, як часний випадок БЧХ-кодів, є\\
    в середньому \((SDif(\mathscr{U}),m,m-t\log{n+1},t)\) - безпечним
    ескізом. (Theorem 6.3 для PinSketch з ресурсу (10))
  \item
    Згідно з теоремою, даний безпечний ескіз має більший рівень
    ентропії, і, відповідно, є більш безпечним для меншої кількості
    перевірочних символів \(t\). У проекті за замовчуванням використано
    32 перевірочні символи, що складає \(6.25\%\) від розміру вхідних
    даних алгоритму.
  \end{itemize}
\end{enumerate}

    \hypertarget{fuzzyextractor-ux432ux438ux43cux43eux433ux438-ux434ux43e-ux432ux445ux456ux434ux43dux438ux445-ux434ux430ux43dux438ux445}{%
\subsubsection{FuzzyExtractor: Вимоги до вхідних
даних}\label{fuzzyextractor-ux432ux438ux43cux43eux433ux438-ux434ux43e-ux432ux445ux456ux434ux43dux438ux445-ux434ux430ux43dux438ux445}}

Для корректної побудови ключа нечітким екстрактором, необхідними є також
початкові умови дo вхідних даних:

\begin{itemize}
\tightlist
\item
  Повна видимість та гарна освітленість обличчя
\item
  Відсутність інших людей на зображеннях
\item
  Бажано монотонний фон та відсутність кольорового освітлення
\end{itemize}

    \hypertarget{ux430ux43dux430ux43bux456ux437-ux435ux444ux435ux43aux442ux438ux432ux43dux43eux441ux442ux456-ux43fux43eux431ux443ux434ux43eux432ux430ux43dux43eux457-ux43cux43eux434ux435ux43bux456}{%
\subsubsection{Аналіз ефективності побудованої
моделі}\label{ux430ux43dux430ux43bux456ux437-ux435ux444ux435ux43aux442ux438ux432ux43dux43eux441ux442ux456-ux43fux43eux431ux443ux434ux43eux432ux430ux43dux43eux457-ux43cux43eux434ux435ux43bux456}}

\begin{itemize}
\tightlist
\item
  Часова складність реалізованого алгоритму (worst-case):
  \(O(n+n^2+nA+B+C)\), де

  \begin{itemize}
  \tightlist
  \item
    n - кількість вхідних зображень
  \item
    A - часова складність алгоритму отримання face-вектору бібліотеки
    dlib.
  \item
    B - часова складність алгоритму SHA512
  \item
    C - часова складність роботи коду Ріда-Соломона
  \end{itemize}
\item
  Просторова складність нечіткого екстрактора: O(n)

  \begin{itemize}
  \tightlist
  \item
    n - кількість вхідних зображень
  \end{itemize}
\item
  Середній час створення ключа на даних, які містять 30 зображень:

  \begin{itemize}
  \tightlist
  \item
    A: \(0.3~-~1.1\) (s) в залежності від розміру зображення
  \item
    B: \(0.1\) (ms)
  \item
    C: \(0.1\) (ms)
  \item
    Алгоритм нечіткого екстрактора: \(10~-~33\) (s)
  \end{itemize}
\item
  Середній час перевірки ключа на даних, які містять 5 зображень:
  \(1.6~-~5.6\) (s)
\item
  Час виконання було протестовано на docker-контейнері, без викорситання
  GPU ресурсів.

  \begin{itemize}
  \tightlist
  \item
    cpu: \texttt{AMD\ Ryzen\ 5/5600H}
  \end{itemize}
\end{itemize}

    \begin{itemize}
\item
  Параметри моделі нечіткого екстрактора:

  \begin{enumerate}
  \def\labelenumi{\arabic{enumi}.}
  \tightlist
  \item
    \(\sigma_0 \in \mathbb{R}^+\setminus \{0\}\) = 0.7
  \item
    \(d \in (0,1)\) = 0.055
  \item
    \(\text{max_unique_hashes} \in \mathbb{N}\cup \{-1\}\) = -1
  \item
    \(\text{p_a_min} \in (0,1)\) = 0.6
  \item
    \(\text{check_symbols_count} \in \mathbb{N}\) = 32
  \item
    \(\text{n_tests} \in \mathbb{N}\) = 250
  \item
    \(\text{sample_size} \in (0,1]\) = 0.7
  \item
    \(\text{min_images} \in \mathbb{N}\) = 5
  \item
    \(\alpha \in (0,1)\) = 0.5
  \end{enumerate}
\end{itemize}

    \hypertarget{ux43fux440ux438ux43aux43bux430ux434ux438-ux432ux438ux43aux43eux440ux438ux441ux442ux430ux43dux43dux44f}{%
\subsubsection{Приклади
використання}\label{ux43fux440ux438ux43aux43bux430ux434ux438-ux432ux438ux43aux43eux440ux438ux441ux442ux430ux43dux43dux44f}}

\begin{itemize}
\tightlist
\item
  Користувач бажає закодувати файл за допомогою алгоритму AES.\\
  Ключ для алгоритму генерується за допомогою класу FuzzyExtractor.
  Оскільки секретний ключ базується на біометричних даних, користувачу
  не потрібно запам'ятовувати пароль.\\
  Натомість для декодування файлу потрібно пройти тест розпізнавання
  обличчя.
\item
  Користувач бажає створити пару ключів для алгоритму ECDSA для
  підписання документа.\\
  Зі сторони сетрифікаційного агентства використовується клас
  FuzzyExtractor.\\
  Тоді можна створити пару ключів на основі face-вектору користувача (та
  послідовності salt). Причому, якщо користувач бажає підписати інший
  документ (без наявності закритого та відкритого ключів), це буде
  набагато простіше, оскільки користувачеві потрібно буде пройти тест
  розпізнавання обличчя для отримання ключа сертифікату.
\item
  Автентифікація користувачів на деякому web-ресурсі. Server-side
  web-ресурсу зберігає послідовності\\
  \(\text{salt}\) та \(\text{check_symbols}\), які є доступними за
  запитом всім користувачам. Клієнт проходить\\
  біометричну аутентифікацію, і отримує ключ, який є спільним секретом
  клієнта і сервера.
\end{itemize}

    \hypertarget{ux456ux43cux456ux442ux430ux446ux456ux439ux43dux456-ux435ux43aux441ux43fux435ux440ux438ux43cux435ux43dux442ux438}{%
\subsubsection{Імітаційні
експерименти}\label{ux456ux43cux456ux442ux430ux446ux456ux439ux43dux456-ux435ux43aux441ux43fux435ux440ux438ux43cux435ux43dux442ux438}}

\begin{itemize}
\tightlist
\item
  Теоретичний розподіл вхідних даних: Теоретичний розподіл зображень є
  рівномірним за значенням face-вектору. Теоретичний розподіл
  послідовності \(\text{salt}\) є рівномірним за значенням кожного
  регістру.
\item
  Алгоритми, які використовувалися для обробки модельованих даних:

  \begin{enumerate}
  \def\labelenumi{\arabic{enumi}.}
  \tightlist
  \item
    Face bounding box (\texttt{dlib} implementation)
  \item
    Face image to face vector mapping (\texttt{face-recignition}
    package)
  \end{enumerate}
\item
  Кількість повторних моделювань для кожного експерименту: порядку 30
  моделювань для кожного набору вхідних даних.
\item
  Узагальнені характеристики якості, які підраховувались:

  \begin{itemize}
  \tightlist
  \item
    Неможливість візуального порівняння вихідних даних ідентичного
    користувача.
  \item
    Значна статистична відстань розподілу ключів від рівномірного для
    корректного набору вхідних даних.
  \item
    Можливість створення та відтворення секретного ключа з використанням
    солі (\(\text{salt}\))
  \item
    Можливість відновлення ключа за різними даними ідентичного
    користувача.
  \item
    Неможливість відновлення ключа одного користувача з використанням
    даних іншого користувача
  \item
    Значення ентропії для послідовності ключа та перевірочних символів
  \item
    Неідентичність вихідних даних алгоритму для різних вхідних даних
    ідентичного користувача
  \end{itemize}
\end{itemize}

Більшість тестових випадків реалізовано у репозиторії за посиланням:
https://github.com/al3xkras/fuzzy-extractors

    \hypertarget{ux43fux440ux438ux43aux43bux430ux434-ux434ux43eux43fux43eux43cux456ux436ux43dux438ux445-ux437ux43dux430ux447ux435ux43dux44c-ux44fux43aux456-ux433ux435ux43dux435ux440ux443ux44eux442ux44cux441ux44f-ux432-ux43fux440ux43eux446ux435ux441ux456-ux440ux43eux431ux43eux442ux438-ux430ux43bux433ux43eux440ux438ux442ux43cux443}{%
\subsubsection{Приклад допоміжних значень, які генеруються в процесі
роботи
алгоритму}\label{ux43fux440ux438ux43aux43bux430ux434-ux434ux43eux43fux43eux43cux456ux436ux43dux438ux445-ux437ux43dux430ux447ux435ux43dux44c-ux44fux43aux456-ux433ux435ux43dux435ux440ux443ux44eux442ux44cux441ux44f-ux432-ux43fux440ux43eux446ux435ux441ux456-ux440ux43eux431ux43eux442ux438-ux430ux43bux433ux43eux440ux438ux442ux43cux443}}

    Розподіл хеш-значень, згенерованих алгоритмом \(\textit{Primary hash}\)
(500 випробувань):

\begin{itemize}
\tightlist
\item
  Для вхідних даних, які задовольняють вимогам (top-5 by frequency):
\item
  Для вхідних даних, які не задовольняють вимогам (top-5 by frequency):
\end{itemize}

Згідно з емпіричними тестами, у розподілі даних, які задовольняють
вимогам деякі елементи спостерігаються із значно більшою ймовірністю.
Якщо дані не задовольняють необхідним вимогам, розподіл ключів є ближчим
до рівномірного розподілу, для подібних даних не було реалізовано
алгоритм створення надійного ключа.

    Приклади пар (\(\text{key},\text{check_symbols}\)), які згенеровано з
використанням нечіткого екстрактора: 1.

\begin{enumerate}
\def\labelenumi{\arabic{enumi}.}
\setcounter{enumi}{1}
\item
\item
\end{enumerate}

Приклади 1. та 3. були згенерованим для однієї людини, у прикладі 2.
було використано біометричні дані іншого користувача. Визначення
належності ключа та перевірочних символів конкретному користувачу є
візуально неможливим.

    Приклади шифрування з використанням солі (SHA512+salt):

\begin{enumerate}
\def\labelenumi{\arabic{enumi}.}
\item
\item
\end{enumerate}

    \hypertarget{ux432ux438ux441ux43dux43eux432ux43aux438}{%
\subsubsection{Висновки}\label{ux432ux438ux441ux43dux43eux432ux43aux438}}

У роботі було реалізовано криптографічно безпечний, ефективний алгоритм
нечіткого екстрактора.\\
Алгоритм генерує приватний ключ базуючись на послідовності зображень або
відеофайлу які відповідають необхідним вимогам. Результатом роботи є
послідовності ключа та перевірочних символів, які для коректно обраних
параметрів моделі є криптографічно безпечними, і можуть бути використані
для ініціалізації інших криптографічних алгоритмів.\\
Було показано, що послідовність перевірочних символів є принаймні
\((SDif(\mathscr{U}),m,m-t\log{n+1},t)\)-безпечною.\\
Значна кількість enterprise-моделей нечітких екстракторів, використовує
коди Ріда-Соломона з малою кількістю перевірочних символів для створення
безпечного ескізу, що також було враховано у роботі, при визначенні
алгоритму та параметрів моделі за замовчуванням.\\
Було протестовано коректність роботи моделі на біометричних даних
ідентичної людини, які отримувалися впродовж 2-х місяців (всього 4
тести, які завершилися успішним відновленням ключа).\\
Позитивними рисами реалізованої моделі нечіткого екстрактора є значна
гнучкість в обранні параметрів, що дозволяє налаштувати нечіткий
екстрактор в залежності від потреб, висока ентропія і можливість
повторного створення неідентичного ключа, який авторизує користувача, у
випадку якщо минулий ключ був опублікований, і можливість використання
послідовності солі для значного ускладнення ряду вразливостей таких як
rainbow table attack, які потенційно дозволяють отримати біометричні
дані користувача.\\
Негативними рисами алгоритму є значна обчислювальна складність створення
ключа зі сторони серверу, що може бути використано зловмисниками для
проведення DoS атак. Тому, використання алгоритму для ініціалізації
web-протоколів є неефективним і потенційно вразливим.\\
Натомість, алгоритм нечіткого екстрактора можна використовувати як
додатковий рівень захисту користувача для дій, які є виключно
конфіденційними (наприклад, створення електронного підпису документу).\\
Для ініціалізації ключа зі сторони серверу, необхідною є персональна
присутність користувача, або надання його даних з довірених ресурсів.
Також користувач може створити ключ локально для ряду задач (наприклад,
аутентифікація у локальній мережі або шифрування файлів). Слід
зауважити, що алгоритм базується на бібліотеці dlib, тому коректна
робота моделі частково залежить від алгоритмів, які були реалізовані у
даній бібліотеці.

    \hypertarget{ux441ux43fux438ux441ux43eux43a-ux432ux438ux43aux43eux440ux438ux441ux442ux430ux43dux438ux445-ux43fux435ux440ux448ux43eux434ux436ux435ux440ux435ux43b}{%
\subsection{Список використаних
першоджерел:}\label{ux441ux43fux438ux441ux43eux43a-ux432ux438ux43aux43eux440ux438ux441ux442ux430ux43dux438ux445-ux43fux435ux440ux448ux43eux434ux436ux435ux440ux435ux43b}}

\begin{itemize}
\item
  \begin{enumerate}
  \def\labelenumi{(\arabic{enumi})}
  \tightlist
  \item
    http://web.cs.ucla.edu/\textasciitilde rafail/PUBLIC/89.pdf\\
    Fuzzy Extractors: How to Generate Strong Keys from Biometrics and
    Other Noisy Data\\
    (Yevgeniy Dodis, Rafail Ostrovsky, Leonid Reyzin, Adam Smith.
    January 20, 2008)
  \end{enumerate}
\item
  \begin{enumerate}
  \def\labelenumi{(\arabic{enumi})}
  \setcounter{enumi}{1}
  \tightlist
  \item
    https://ro.uow.edu.au/cgi/viewcontent.cgi?article=1698\&context=eispapers1\\
    LI, N., Guo, F., Mu, Y., Susilo, W. \& Nepal, S. (2017). Fuzzy
    Extractors for Biometric Identification. 37th IEEE\\
    Internaitonal Conference on Distributed Computing Systems (ICDCS
    2017) (pp.~667-677). United States: IEEE.
  \end{enumerate}
\item
  \begin{enumerate}
  \def\labelenumi{(\arabic{enumi})}
  \setcounter{enumi}{2}
  \tightlist
  \item
    https://www.cs.bu.edu/\textasciitilde reyzin/papers/fuzzysurvey.pdf\\
    Fuzzy Extractors. A Brief Survey of Results from 2004 to 2006\\
    (Yevgeniy Dodis, Leonid Reyzin, Adam Smith)
  \end{enumerate}
\item
  \begin{enumerate}
  \def\labelenumi{(\arabic{enumi})}
  \setcounter{enumi}{3}
  \tightlist
  \item
    https://faculty.math.illinois.edu/\textasciitilde duursma/CT/RS-1960.pdf\\
    POLYNOMIAL CODES OVER CERTAIN FINITE FIELDS* I. S. REED AND G.
    SOLOMON
  \end{enumerate}
\item
  \begin{enumerate}
  \def\labelenumi{(\arabic{enumi})}
  \setcounter{enumi}{4}
  \tightlist
  \item
    https://www.arijuels.com/wp-content/uploads/2013/09/JS02.pdf\\
    A Fuzzy Vault Scheme (Ari Juels and Madhu Sudan) RSA Laboratories,
    Bedford, MA 01730, USA
  \end{enumerate}
\item
  \begin{enumerate}
  \def\labelenumi{(\arabic{enumi})}
  \setcounter{enumi}{5}
  \tightlist
  \item
    https://digital.csic.es/bitstream/10261/15966/1/SAM3262.pdf\\
    Biometric Fuzzy Extractor Scheme for Iris Templates\\
    F. Hernandez Alvarez, L. Hernandez Encinas, C. Sanchez Avila\\
    Departamento Matematica Aplicada a las Tecnolog ıas de la
    Informacion, E.T.S.I.T.\\
    Universidad Politecnica de Madrid. Spain.
  \end{enumerate}
\item
  \begin{enumerate}
  \def\labelenumi{(\arabic{enumi})}
  \setcounter{enumi}{6}
  \tightlist
  \item
    https://nure.ua/wp-content/uploads/2018/Scientific\_editions/are\_2018\_24.pdf\\
    ПОРІВНЯЛЬНИЙ АНАЛІЗ БІОМЕТРИЧНИХ КРИПТОСИСТЕМ\\
    М. С. ЛУЦЕНКО, О. О. КУЗНЕЦОВ, Д. І. ПРОКОПОВИЧ-ТКАЧЕНКО, В. П.
    ЗВЄРЄВ, А. О. УВАРОВА
  \end{enumerate}
\end{itemize}


    % Add a bibliography block to the postdoc
    
    
    
\end{document}
